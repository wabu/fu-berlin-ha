\documentclass[a4paper,10pt]{report}

\topmargin -3cm
%\topskip0cm
%\footskip0cm
%\headsep0cm
\parindent0cm
\oddsidemargin -1cm
\evensidemargin -1cm
\headheight 3cm
\textheight 23cm
\textwidth 18cm

\author{Daniel W\"aber (4049590)}
\title{\"Ubung}

\usepackage{ucs}
\usepackage[utf8x]{inputenc}
\usepackage{german}
\usepackage{color}

\pagestyle{empty}
\usepackage{makeidx}
\usepackage{amsmath}
\usepackage{amsfonts}
\usepackage{amssymb,euscript}
\usepackage{dsfont}
\usepackage{listings}
\usepackage{enumerate}
\newfont{\Fr}{eufm10}
\newfont{\Sc}{eusm10}
\newfont{\Bb}{msbm10}
\newcommand{\limin}{\lim_{n\rightarrow\infty}}
\newcommand{\limix}{\lim_{x\rightarrow\infty}}
\newcommand{\limun}{\lim_{n\rightarrow -\infty}}
\newcommand{\limux}{\lim_{n\rightarrow -\infty}}
\newcommand{\limx}{\lim_{x\rightarrow x_0}}
\newcommand{\limh}{\lim_{h\rightarrow 0}}
\newcommand{\defi}{\paragraph{Definition:}}
\newcommand{\bew}{\paragraph{Beweis:}}
\newcommand{\satz}{\paragraph{Satz:}}
\newcommand{\bsp}{\paragraph{Beispiel:}}
\newcommand{\lemma}{\paragraph{Lemma:}}
\newcommand{\N}{\mathds{N}}
\newcommand{\Z}{\mathds{Z}}
\newcommand{\Q}{\mathds{Q}}
\newcommand{\R}{\mathds{R}}
\newcommand{\C}{\mathds{C}}
\newcommand{\K}{\mathds{K}}
\newcommand{\A}{\mathds{A}}
\newcommand{\qed}{$\hfill\blacksquare$}
\newcommand{\arsinh}{\operatorname{arsinh} }
\newcommand{\arcosh}{\operatorname{arcosh} }
\newcommand{\wP}{\mathcal{P} }
\newcommand{\gdw}{$\Leftrightarrow$}
\newcommand{\tf}{$\Rightarrow$}
\newcommand{\mgdw}{\Leftrightarrow}
\newcommand{\mtf}{\Rightarrow}
\newcommand{\Bild}{\text{Bild}}
\newcommand{\Kern}{\text{kern}}
\newcommand{\rg}{\text{rg}}
\newcommand{\deff}{\text{deff}}

\newcommand{\alphato}{\underset{\alpha}\to}
\newcommand{\betato}{\underset{\beta}\to}
\newcommand{\etato}{\underset{\eta}\to}
\newcommand{\ito}{\underset{i}\to}
\newcommand{\sto}{\underset{s}\to}
\newcommand{\kto}{\underset{k}\to}
\newcommand{\xto}{\underset{x}\to}

\usepackage{fancyhdr}
\pagestyle{fancy}
\lhead{Daniel Waeber\\Michael Kmoch}
\chead{"Ubungsblatt \nr\\\today}
\rhead{HA\\Tutor: Claudia Dieckmann}



\newcommand{\nr}{7}

\begin{document}
\section*{Aufgabe 2}
\paragraph{Korrektheit}
Annahme:
Sei $u,v$ Knoten, die nicht in der richtigen Reihenfolge ausgegeben wurden, dh. eine Kante $k = (v,u)$ in $E$ existiert,
dennoch $u$ vor $v$ ausgegeben wird.

Da $u$ nur ausgegeben wird, wenn es einen Ingrad von 0
besitzt, muesste also $k$ erst entfernt worden sein, bevor $u$ ausgegben wird.
Allerdings wird eine Kante nur entfernt, wenn der Ursprungsknoten
ausgegeben wurde, was der Annahme widerspricht, dass $u$ vor $v$ ausgegeben wurden

\section*{Aufgabe 3}
\begin{itemize}
\paragraph{Datenstrukturen}
\item zum Speichern des Graphen wird eine Adjuszenzliste verwendet
\item zum Finden der Kante minimalen Gewichtes benoetigt man eine Hilfsstruktur:
    wir verwenden eine Liste, in der fuer jeden Knoten aud $V$ gespeichert wird,
    wie teuer dieser ueber eine an $T$ anschliesende Kante erreichbar ist, bzw. 
    falls dieser nicht erreichbar ist, der Wert $\infty$. Zusaetzlich wird in der Liste
    der andere Endknoten, ueber den dieser Knoten erreichbar ist, gespeichert.
    $ \to $ Initialisierung $O(|V|)$
\item Wird ein Knoten zu $T$ hinzugefuegt, werden entsprechend die Werte der ueber diesen Knoten 
    erreichbaren Knoten erneuert $ \to $ Laufzeit $O(|V|)$
\item zum Findern einer Kante kann nun diese Liste nach dem minimalen wert Durchsucht werden.
     $ \to $ Laufzeit $O(|V|)$
\end{itemize}

\paragraph{Laufzeit}
In jedem der $|V|-1$ Schritte betraegt die Laufzeit $O(|V|)$, also ergibt sich eine Laufzeit von 
$O(|V|^2)$

\paragraph{Korrektheit}
$T$ ist ein Baum, da nur Kanten, die auf Knoten meit einem Endknoten ausserhalb von $W$ hinzugefuegt werden.
Sei nun $Y$ der minimale aufpannende Baum von $G$. 

Falls nun $Y \neq T$, dann muesste eine Kante $k$ zu $T$ hinzugefuegt worden sein, die nicht in $Y$ ist.
Man bezeichne die Knoten in $T$ vor der hinzufuegung von $k$ mit $W$. Da $Y$ ein MST ist, gibt es einen Pfad in $Y$,
der die Endknoten von $k$ verbindet. Auf diesem Pfad muss es eine Kante $k'$ geben, die einen Knoten in $W$ mit einem
Knoten ausserhalb $W$ verbindet, da einer der Endknoten von $k$ in $V$ liegt, der andere ausserhalb.
Da $Y$ ein MST mussen die Kosten fuer den Pfad minimal sein, also auch geringer als die Kosten von $w(k)$.
Da der Pfad aber $k'$ einschliest, muessen auch die Kosten von $w(k')$ geringer sein als die von $w(k)$.
Da $k'$ auch einen Knoten innerhalb von $V$ mit einem Knoten ausserhalb von $V$ verbindet, haette der Algorithmus
die Kante $k'$ auswaehlen muessen, da die Kosten geringer sind als die von $k$.

Also muss $T = Y$ sein.

\end{document}
