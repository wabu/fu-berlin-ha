\documentclass[a4paper,10pt]{report}

\topmargin -3cm
%\topskip0cm
%\footskip0cm
%\headsep0cm
\parindent0cm
\oddsidemargin -1cm
\evensidemargin -1cm
\headheight 3cm
\textheight 23cm
\textwidth 18cm

\author{Daniel W\"aber (4049590)}
\title{\"Ubung}

\usepackage{ucs}
\usepackage[utf8x]{inputenc}
\usepackage{german}
\usepackage{color}

\pagestyle{empty}
\usepackage{makeidx}
\usepackage{amsmath}
\usepackage{amsfonts}
\usepackage{amssymb,euscript}
\usepackage{dsfont}
\usepackage{listings}
\usepackage{enumerate}
\newfont{\Fr}{eufm10}
\newfont{\Sc}{eusm10}
\newfont{\Bb}{msbm10}
\newcommand{\limin}{\lim_{n\rightarrow\infty}}
\newcommand{\limix}{\lim_{x\rightarrow\infty}}
\newcommand{\limun}{\lim_{n\rightarrow -\infty}}
\newcommand{\limux}{\lim_{n\rightarrow -\infty}}
\newcommand{\limx}{\lim_{x\rightarrow x_0}}
\newcommand{\limh}{\lim_{h\rightarrow 0}}
\newcommand{\defi}{\paragraph{Definition:}}
\newcommand{\bew}{\paragraph{Beweis:}}
\newcommand{\satz}{\paragraph{Satz:}}
\newcommand{\bsp}{\paragraph{Beispiel:}}
\newcommand{\lemma}{\paragraph{Lemma:}}
\newcommand{\N}{\mathds{N}}
\newcommand{\Z}{\mathds{Z}}
\newcommand{\Q}{\mathds{Q}}
\newcommand{\R}{\mathds{R}}
\newcommand{\C}{\mathds{C}}
\newcommand{\K}{\mathds{K}}
\newcommand{\A}{\mathds{A}}
\newcommand{\qed}{$\hfill\blacksquare$}
\newcommand{\arsinh}{\operatorname{arsinh} }
\newcommand{\arcosh}{\operatorname{arcosh} }
\newcommand{\wP}{\mathcal{P} }
\newcommand{\gdw}{$\Leftrightarrow$}
\newcommand{\tf}{$\Rightarrow$}
\newcommand{\mgdw}{\Leftrightarrow}
\newcommand{\mtf}{\Rightarrow}
\newcommand{\Bild}{\text{Bild}}
\newcommand{\Kern}{\text{kern}}
\newcommand{\rg}{\text{rg}}
\newcommand{\deff}{\text{deff}}

\newcommand{\alphato}{\underset{\alpha}\to}
\newcommand{\betato}{\underset{\beta}\to}
\newcommand{\etato}{\underset{\eta}\to}
\newcommand{\ito}{\underset{i}\to}
\newcommand{\sto}{\underset{s}\to}
\newcommand{\kto}{\underset{k}\to}
\newcommand{\xto}{\underset{x}\to}

\usepackage{fancyhdr}
\pagestyle{fancy}
\lhead{Daniel Waeber\\Michael Kmoch}
\chead{"Ubungsblatt \nr\\\today}
\rhead{HA\\Tutor: Claudia Dieckmann}



\newcommand{\nr}{6}

\begin{document}
\section*{Aufgabe 1}
$p_1=p_2=0.2; p_3=0; p_4=0.1; q_0=0.2; q_1 = q_2=0; q_3=0.2; q_4=0.1$

\begin{tabular}[h]{r|lllll}
$i$   & 0 & 1 & 2 & 3 & 4 \\\hline
      & $w_{1,0} = 0.2$ & $w_{2,1}=0.0$ & $w_{3,2}=0.0$ & $w_{4,3}=0.2$ & $w_{5,4}=0.1$ \\
      & $c_{1,0} = 0.0$ & $c_{2,1}=0.0$ & $c_{3,2}=0.0$ & $c_{4,3}=0.0$ & $c_{5,4}=0.0$ \\
\hline
$k=0$ &                 & $r_{1,1}=1$   & $r_{2,2}=2$   & $r_{3,3}=3$   & $r_{4,4}=4$   \\
      &                 & $w_{1,1}=0.4$ & $w_{2,2}=0.2$ & $w_{3,3}=0.2$ & $w_{4,4}=0.4$ \\
      &                 & $c_{1,1}=0.4$ & $c_{2,2}=0.2$ & $c_{3,3}=0.2$ & $c_{4,4}=0.4$ \\
\hline
$k=1$ &                 &               & $r_{1,2}=2$   & $r_{2,3}=2$   & $r_{3,4}=3$   \\
      &                 &               & $w_{1,2}=0.6$ & $w_{2,3}=0.4$ & $w_{3,4}=0.4$ \\
      &                 &               & $c_{1,2}=1.0$ & $c_{2,3}=0.6$ & $c_{3,4}=0.8$ \\
\hline
$k=2$ &                 &               &               & $r_{1,3}=2$   & $r_{2,4}=4$   \\
      &                 &               &               & $w_{1,3}=0.8$ & $w_{2,4}=0.6$ \\
      &                 &               &               & $c_{1,3}=1.4$ & $c_{2,4}=1.2$ \\
\hline
$k=3$ &                 &               &               &               & $r_{1,4}=1$   \\
      &                 &               &               &               & $w_{1,4}=1.0$ \\
      &                 &               &               &               & $c_{1,4}=2.2$ \\
\end{tabular}

\begin{dot2tex}[autosize,options=-tmath]
graph {
    1 -- b0
    b0[label="]-\infty,1[",shape=none]
    1 -- 4

    4 -- 2
    4 -- b4
    b4[label="]4,+\infty[",shape=none]

    2 -- b1
    b1[label="]1,2[",shape=none]
    2 -- 3

    3 -- b2
    b2[label="]2,3[",shape=none]
    3 -- b3
    b3[label="]3,4[",shape=none]
}
\end{dot2tex}

\section*{Aufgabe 2a)}
\paragraph{Idee}
Zur Bestimmung von $r_{i,j}$ muss also nicht jede Wurzel betrachtet werden, es reicht aus,
zwischen den Wurzeln in der Tabelle oberhalb zu suchen.

\paragraph{Algorithmus}
\begin{alltt}
for \(i=0\) to \(n\) do
  \(w\sb{i+1,i} = q\sb{i}\)
  \(c\sb{i+1,i} = 0  \)
  \(r\sb{i+1,i} = i \)
for \(k=0\) to \(n-1\) do
  for \(i=1\) to \(n-k\) do
    \(j = i+k\)
    \(m = i\)
    \(c = \infty\)
    for \(n = r\sb{i,j-1}\) to \(r\sb{i+1,j}\) do
      \(s = c\sb{i,n-1} + c\sb{n+1,j}\)
      if \(s < c\) then
        \(c = s\)
        \(m = n\)
    \(r\sb{i,j} = m\)
    \(w\sb{i,j} = w\sb{i,m-1} + w\sb{m+1,j} + p\sb{m}\)
    \(c\sb{i,j} = c\sb{i,m-1} + c\sb{m+1,j} + w\sb{i,j}\)
\end{alltt}

\paragraph{Analyse} $ $

Die innere Schleif ist genauer zu betrachten. Die restlichen Zeilen
werden maxiaml $n^2$-mal durchlaufen, da sie innerhalb der 2. Schelife stehen.

Man betrachtet die Start- und End-Indizes der inneren Scheife bei einem Durchlauf
ueber $i$ in der 2. Schleife fuer ein festes $k$: Schleifenanfang liegt bei $r_{i,i+k-1}$ und das Ende bei $r_{i+1,i+k}$.

Das Ende $r_{i+1,i+k}$ fuer ein $i$ faellt also mit dem Anfang fuer $i+1$ zusammen.
Also wird bei dem Durchlauf ueber $i = 1 \text{ to } n-k$ jeder moegliche Wurzelwert 
hoechstens zwei mal durchlaufen.
Also werden die Anweisungen in der innersten Schleife auch nur maximal 2 mal
fuer jeden Durchlauf der 2. Schleife ausgefuehrt.

Die Laufzeit betraegt also insgesamt $O(n^2)$
\section*{Aufgabe 2b)}

\begin{eqnarray}
T(1) &=& c\\
T(n) &=& \sum_{i=1}^{n-1} T(i) + T(n-i) \\
     &=& 2 \sum_{i=1}^{n-1} T(i) \\
     &=& 2 \sum_{i=1}^{n-2} T(i) + 2 \cdot T(n-1) \\
     &=& 2 \sum_{i=1}^{n-2} T(i) + 2 \sum_{i=1}^{n-2} T(i) \\
     &=& 4 \sum_{i=1}^{n-2} T(i) \\
     &=& \cdots \\
     &=& 2^k \sum_{i=1}^{n-k} T(i) \\ 
     &=& \cdots \\
     &=& 2^{n-1} \sum_{i=1}^{1} T(i) \\ 
     &=& 2^{n-1} c \\
     &=& O(2^n)
\end{eqnarray}

\end{document}
