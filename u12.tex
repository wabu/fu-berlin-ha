\documentclass[a4paper,10pt]{report}

\topmargin -3cm
%\topskip0cm
%\footskip0cm
%\headsep0cm
\parindent0cm
\oddsidemargin -1cm
\evensidemargin -1cm
\headheight 3cm
\textheight 23cm
\textwidth 18cm

\author{Daniel W\"aber (4049590)}
\title{\"Ubung}

\usepackage{ucs}
\usepackage[utf8x]{inputenc}
\usepackage{german}
\usepackage{color}

\pagestyle{empty}
\usepackage{makeidx}
\usepackage{amsmath}
\usepackage{amsfonts}
\usepackage{amssymb,euscript}
\usepackage{dsfont}
\usepackage{listings}
\usepackage{enumerate}
\newfont{\Fr}{eufm10}
\newfont{\Sc}{eusm10}
\newfont{\Bb}{msbm10}
\newcommand{\limin}{\lim_{n\rightarrow\infty}}
\newcommand{\limix}{\lim_{x\rightarrow\infty}}
\newcommand{\limun}{\lim_{n\rightarrow -\infty}}
\newcommand{\limux}{\lim_{n\rightarrow -\infty}}
\newcommand{\limx}{\lim_{x\rightarrow x_0}}
\newcommand{\limh}{\lim_{h\rightarrow 0}}
\newcommand{\defi}{\paragraph{Definition:}}
\newcommand{\bew}{\paragraph{Beweis:}}
\newcommand{\satz}{\paragraph{Satz:}}
\newcommand{\bsp}{\paragraph{Beispiel:}}
\newcommand{\lemma}{\paragraph{Lemma:}}
\newcommand{\N}{\mathds{N}}
\newcommand{\Z}{\mathds{Z}}
\newcommand{\Q}{\mathds{Q}}
\newcommand{\R}{\mathds{R}}
\newcommand{\C}{\mathds{C}}
\newcommand{\K}{\mathds{K}}
\newcommand{\A}{\mathds{A}}
\newcommand{\qed}{$\hfill\blacksquare$}
\newcommand{\arsinh}{\operatorname{arsinh} }
\newcommand{\arcosh}{\operatorname{arcosh} }
\newcommand{\wP}{\mathcal{P} }
\newcommand{\gdw}{$\Leftrightarrow$}
\newcommand{\tf}{$\Rightarrow$}
\newcommand{\mgdw}{\Leftrightarrow}
\newcommand{\mtf}{\Rightarrow}
\newcommand{\Bild}{\text{Bild}}
\newcommand{\Kern}{\text{kern}}
\newcommand{\rg}{\text{rg}}
\newcommand{\deff}{\text{deff}}

\newcommand{\alphato}{\underset{\alpha}\to}
\newcommand{\betato}{\underset{\beta}\to}
\newcommand{\etato}{\underset{\eta}\to}
\newcommand{\ito}{\underset{i}\to}
\newcommand{\sto}{\underset{s}\to}
\newcommand{\kto}{\underset{k}\to}
\newcommand{\xto}{\underset{x}\to}

\usepackage{fancyhdr}
\pagestyle{fancy}
\lhead{Daniel Waeber\\Michael Kmoch}
\chead{"Ubungsblatt \nr\\\today}
\rhead{HA\\Tutor: Claudia Dieckmann}



\newcommand{\nr}{12}

\begin{document}
\section*{Aufgabe 1}
\paragraph{a) Hamilton-Pfad $\leq_p$ Hamilton-Kreis}

\begin{eqnarray}
f &:& \text{Menge unger. Graphen} \to \text{Menge ungr. Graphen}\\
  & & G \mapsto G'
\end{eqnarray}

Dabei wird $G$ mit Knotenmenge $V$ und Kantenmenge $E$ auf $G'$ abgebildet, mit 
$V' = V \cup \{\tilde{v}\}$ und $E' = E \cup \{ (\tilde{v}, v), v\in~V\}$.

Da das hinzufuegen des Knoten in konstanter, das hinzufuegen der neuen Kanten in lineare Zeit
geht, ist $f$ in polynomieller Zeit berechenbar.

\paragraph{$G$ hat Hamilton-Pfad $\Rightarrow$ $G'$ hat Hamilton-Kreis} $ $ 

Da $G$ Ham-Pfad, $\exists v_1 \cdots v_n$, sodass $(v_i, v_i+1) \in E$ fuer $i=1\cdots n-1$
und $\{v_1, \cdots, v_n\} = V$. 

Es gilt nun auch, $(v_i, v_i+1) \in E'$ und $(\tilde{v}, v_n), (\tilde{v}, v_1) \in E'$
und $\{v_1, \cdots, v_n, \tilde{v}\} = V'$. Also ist $v_1 \cdots v_n \tilde{v}$ Hamilton-Kreis in $G'$

\paragraph{$G$ hat Hamilton-Pfad $\Leftarrow$ $G'$ hat Hamilton-Kreis} $ $ 

$G'$ hat Ham-Kreis \tf $\exists \tilde{v} v_1 \cdots v_n$, 
    sodass $(v_i, v_i+1) \in E'$ fuer $i=1\cdots n$ und $(\tilde{v}, v_n), (\tilde{v}, v_1) \in E'$ und $\{v_1, \cdots, v_n, \tilde{v}\} = V'$.

Es gilt auch, $(v_i, v_i+1) \in E$ fuer $i=1\cdots n$, und $\{v_1, \cdots, v_n\} = V$. Also
ist $v_1 \cdots v_n$ Hamilton-Pfad in $G$.



\paragraph{b) Hamilton-Pfad $\leq_p$ TGI}

\begin{eqnarray}
f &:& \text{Menge unger. Graphen} \to \{\langle G_1, G_2 \rangle | \text{$G_1, G_2$ unger. Graphen}\}\\
  & & G \mapsto \langle G_1, G_2 \rangle
\end{eqnarray}

Dabei wird $G = (V,E)$ abgebildet auf $\langle G_1, G_2 \rangle$ mit
$V_1 = \{w_1,w_2, \cdots, w_n\}$ mit $n=|V|$, $E_1 = \{(w_i, w_{i+1}), i=1\cdots n-1\}$, $V_2 = V$, $E_2 = E$

Die Konstroktion von $G_1$ und das kopieren des Graphfen $G$ ist jeweils in lineare Zeit moeglich.

\paragraph{$G$ hat Hamilton-Pfad $\Rightarrow$ $G_2$ besitzt Teilgraph $G_2'$, sodass $G_1$ isomorph $G_2'$} $ $ 

Da $G$ Ham-Pfad, $\exists v_1 \cdots v_n$, sodass $(v_i, v_i+1) \in E$ fuer $i=1\cdots n-1$
und $\{v_1, \cdots, v_n\} = V$. 

Es gilt $E_2' = \{(v_i, v_i+1), i = 1 \cdots n-1\} \subseteq E = E_2'$ und $\{v_1, \cdots, v_n\} = V = V_2$,
also ist $G_2' = (V_2, E_2)$ ist Teilgraph von $G_2$. Durch $f: v_i \mapsto w_i$ fuer $i=1\cdots n$
ist nun der Isomorphismus definiert, der $G_2'$ auf $G_1$ abbildet.

\paragraph{$G$ hat Hamilton-Pfad $\Leftarrow$ $G_2$ besitzt Teilgraph $G_2'$, sodass $G_1$ isomorph $G_2'$} $ $ 

$G_1 = (V_1, \{(w_i, w_i+1) \})$ ist Isomorph zu $G_2'$. Also exisiterit $v_1 \cdots v_n \in V_2'$,
    mit $(v_i, v_i+1) \in E_2'$. da $|V_1| = n$ muss auch $|V_2'| = n = |V_2|$, also ist $V_2' = V_2$.
    Somit ist $v_1 \cdots v_n$ ein Hamliton-Pfad in $V_2$. Da $G_2 = G$ ist dieser Pfad somit auch Hamilton-Pfad 
    in $G$.

\section*{Aufgabe 2}
\paragraph{a) Subset-Sum $\leq_p$ Partition} $ $
\begin{eqnarray}
f &:& \N^* \times \N \to \N^*\\
  & & \langle m_1,k \cdots, m_n, k \rangle \mapsto n_1, \cdots, n_n, n_{n+1}
\end{eqnarray}
Sei $m = \sum_{i= 1\ldots n} m_i$ und $s = m-2k$. 
$n_i = m_i$ fuer $i=1\cdots n$, $n_{n+1} = s$

Da die Berechnung der Summe in linearer Zeit moeglich ist, ist $f$ auch in linearer Zeit berechenbar.


\paragraph{Subset-Sum $\Rightarrow$ Partition} $ $

Sei $I$ die Indexmenge fuer die gilt, dass $\sum_{i\in I} m_i = k$,
sei $I' = I \cup \{n+1\}$.

Es gilt $\sum_{i\in I} n_i = k + s = k+m-2k = m-k$ und $\sum_{i\in \bar{I}} n_i = \sum_{i=1}^{n} i - \sum_{i\in I} = m-k$

Also kann ein Partition mit gleicher Summe gefunden werden.


\paragraph{Subset-Sum $\Leftarrow$ Partition} $ $

Sei $I$ die Indexmenge, sodass $\sum_{i\in I} n_i = \sum_{i\in\bar{I}} n_i$.
Obda ist $n+1 \in I$.

Es gilt 
\begin{eqnarray}
\sum_{i\in \bar{I}} n_i &=& \sum_{i\in I} n_i \\
\sum_{i\in \bar{I}} n_i &=&  s + \sum_{i\in I \setminus \{n+1\}} n_i \\
 s &=& \sum_{i\in \bar{I}} n_i - \sum_{i \in I \setminus \{n+1\}} n_i \\
   &=& \sum_{i\in \bar{I}} n_i + \sum_{i \in I \setminus \{n+1\}} n_i - 2 \cdot  \sum_{i \in I \setminus \{n+1\}} n_i \\
   &=& \sum_{i=1}^n n_i - 2 \cdot \sum_{i\in I \setminus \{n+1\}} n_i \\ 
 m-2k &=& m - 2 \cdot \sum_{i\in I \setminus \{n+1\}} n_i \\ 
 k &=& \sum_{i\in I\setminus \{n+1\}} i \\
\end{eqnarray}

Also erfuellt $\bar{I}$ das Subset-Sum Problem fuer $M,k$




\begin{eqnarray}
f &:& \N^* \to \N^* \\
  & & a_1,\cdots,a_n \mapsto \bar{l_1}, \bar{l_2}, l_1, \cdots,l_n, \dot{l_2}, \dot{l_1}, l
\end{eqnarray}
wobei $\bar{l_1} = \dot{l_1} = A$, $\bar{l_2} = \dot{l_2} = \frac{A}{2}$, $l_i = a_i$ fuer $i=1 \cdots n$,
      $l=A$

\paragraph{Partition $\Rightarrow$ Zollstock} $ $

Sei $I$ die Indexmenge, sodass $\sum_{i\in I} a_i = \sum_{i\in\bar{I}} a_i$.

Es wird $\bar{l_1}$ und $\dot{l_1}$ nach rechts, $\bar{l_2}$ und $\dot{l_2}$ nach links gefaltet.
Nun kann der Zollstock so gefaltet werden,
dass alle Elemente in $I$ nach links, alle Elemente in $\bar{I}$ nach rechts gefaltet werden.

Anfang und Ende des inneren Teils $l_1 \cdots l_n$ faellt auf den gleichen punkt, 
da $\sum_{i\in I} a_i = \sum_{i\in\bar{I}} a_i$.

Des weitern gilt dass $\sum_{i\in I} a_1 = \frac{A}{2} = \sum_{i\in \bar{I}} a_i$, 
also kann der inner Teil hoechstens eine Laenge von $\frac{A}{2}$ besitzen.

Wegen dem erste Teil $\bar{l_1}$ muss der Zollstock eine Laenge $\geq A$ haben.
Da $\bar{l_2}$ bei $\frac{A}{2}$ aufhoert, und der Innere teil von $\frac{A}{2}$ begrenzt ist,
ragt dieser Teil auch nicht ueber $\bar{l_1}$ herraus. 
$\dot{l_2}$ setzt wieder bei $\frac{A}{2}$ an, sodass sich der letzte Teil ueber den Anfang legt.

Also kann der Zollstock zur Laenge $A$ gefaltet werden.

\paragraph{Partition $\Leftarrow$ Zollstock} $ $

Sei $I$ die Indexmenge der inneren Elemente, die nach rechts gefaltet ist,
    $\bar{I}$ die Indexmenge der inneren Elemente, die nach links gefaltet ist,

Obda ist $\bar{l_1}$ nach rechts gefaltet. $\bar{l_2}$ muss nach links gefaltet sein,
da dieses sonst den Zollstock ueber $A$ hinaus verlaengern wuerde.

$\dot{l_2}$ muss wieder bei $\frac{A}{2}$ ansetzen, damit dessen Ende bei $A$ oder $0$ liegen kann.
Ansonsten koennte das letzte Element $\dot{l_1}$ nicht so angesetzt werden, dass es nicht ueber $A$ hinausragt.

Also liegt Anfang und Ende des innern Teils auf $\frac{A}{2}$. Daraus folgt, dass
$\sum_{i\in I} l_i = \sum_{i\in \bar{I}} l_i$.

Also ist fuer die Indexmenge $I$ Partition erfuellt.




\section*{Aufgabe 3}


\end{document}
