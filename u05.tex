\documentclass[a4paper,10pt]{report}

\topmargin -3cm
%\topskip0cm
%\footskip0cm
%\headsep0cm
\parindent0cm
\oddsidemargin -1cm
\evensidemargin -1cm
\headheight 3cm
\textheight 23cm
\textwidth 18cm

\author{Daniel W\"aber (4049590)}
\title{\"Ubung}

\usepackage{ucs}
\usepackage[utf8x]{inputenc}
\usepackage{german}
\usepackage{color}

\pagestyle{empty}
\usepackage{makeidx}
\usepackage{amsmath}
\usepackage{amsfonts}
\usepackage{amssymb,euscript}
\usepackage{dsfont}
\usepackage{listings}
\usepackage{enumerate}
\newfont{\Fr}{eufm10}
\newfont{\Sc}{eusm10}
\newfont{\Bb}{msbm10}
\newcommand{\limin}{\lim_{n\rightarrow\infty}}
\newcommand{\limix}{\lim_{x\rightarrow\infty}}
\newcommand{\limun}{\lim_{n\rightarrow -\infty}}
\newcommand{\limux}{\lim_{n\rightarrow -\infty}}
\newcommand{\limx}{\lim_{x\rightarrow x_0}}
\newcommand{\limh}{\lim_{h\rightarrow 0}}
\newcommand{\defi}{\paragraph{Definition:}}
\newcommand{\bew}{\paragraph{Beweis:}}
\newcommand{\satz}{\paragraph{Satz:}}
\newcommand{\bsp}{\paragraph{Beispiel:}}
\newcommand{\lemma}{\paragraph{Lemma:}}
\newcommand{\N}{\mathds{N}}
\newcommand{\Z}{\mathds{Z}}
\newcommand{\Q}{\mathds{Q}}
\newcommand{\R}{\mathds{R}}
\newcommand{\C}{\mathds{C}}
\newcommand{\K}{\mathds{K}}
\newcommand{\A}{\mathds{A}}
\newcommand{\qed}{$\hfill\blacksquare$}
\newcommand{\arsinh}{\operatorname{arsinh} }
\newcommand{\arcosh}{\operatorname{arcosh} }
\newcommand{\wP}{\mathcal{P} }
\newcommand{\gdw}{$\Leftrightarrow$}
\newcommand{\tf}{$\Rightarrow$}
\newcommand{\mgdw}{\Leftrightarrow}
\newcommand{\mtf}{\Rightarrow}
\newcommand{\Bild}{\text{Bild}}
\newcommand{\Kern}{\text{kern}}
\newcommand{\rg}{\text{rg}}
\newcommand{\deff}{\text{deff}}

\newcommand{\alphato}{\underset{\alpha}\to}
\newcommand{\betato}{\underset{\beta}\to}
\newcommand{\etato}{\underset{\eta}\to}
\newcommand{\ito}{\underset{i}\to}
\newcommand{\sto}{\underset{s}\to}
\newcommand{\kto}{\underset{k}\to}
\newcommand{\xto}{\underset{x}\to}

\usepackage{fancyhdr}
\pagestyle{fancy}
\lhead{Daniel Waeber\\Michael Kmoch}
\chead{"Ubungsblatt \nr\\\today}
\rhead{HA\\Tutor: Claudia Dieckmann}



\newcommand{\nr}{5}

\begin{document}
\section*{Aufgabe 1}
\begin{enumerate}[(a)]
\item AVL-Baum

\begin{itemize}
\item 32: 
\begin{dot2tex}[autosize]
graph G {
    32
}
\end{dot2tex}

\item 27: 
\begin{dot2tex}[autosize]
graph G {
    32 -- 27
    0[style=invis]
    32 -- 0[style=invis]
}
\end{dot2tex}

\item 13: AVL fuer Wurzel verletzt, einfache Rotation noetig
\begin{dot2tex}[autosize]
graph G {
    a32[label=32,shape=octagon]
    a27[label=27]
    a13[label=13]
    a32 -- a27 -- a13
    a0[style=invis]
    a32 -- a0[style=invis]
    a27 -- a0[style=invis]
    27 -- 13
    27 -- 32
}
\end{dot2tex}

\item 41:
\begin{dot2tex}[autosize]
graph G {
    0[style=invis]
    27 -- 13
    27 -- 32
    32 -- 0[style=invis]
    32 -- 41
}
\end{dot2tex}

\item 55: Rotation in Teilbaum noetig
\begin{dot2tex}[autosize]
graph G {
    0[style=invis]
    27 -- 13
    27 -- 32
    32[shape=octagon]
    32 -- 0[style=invis]
    32 -- 41
    41 -- 0[style=invis]
    41 -- 55

    a27[label=27]
    a13[label=13]
    a32[label=32]
    a41[label=41]
    a55[label=55]

    a27 -- a13
    a27 -- a41
    a41 -- a32
    a41 -- a55
}
\end{dot2tex}

\item 86: Doppelrotation in Wurzel noetig
\begin{dot2tex}[autosize]
graph G {
    a0[style=invis]
    a27[label=27,shape=octagon]
    a13[label=13]
    a32[label=32]
    a41[label=41]
    a55[label=55]
    a86[label=86]

    a27 -- a13
    a27 -- a41
    a41 -- a32
    a41 -- a55
    a55 -- a0[style=invis]
    a55 -- a86

    0[style=invis]
    41 -- 27
    27 -- 13
    27 -- 32
    41 -- 55
    55 -- 0[style=invis]
    55 -- 86
}
\end{dot2tex}
\item 72: Doppel-Rotation noetig
\begin{dot2tex}[autosize]
graph G {
    41 -- 27
    27 -- 13
    27 -- 32
    41 -- 55
    0[style=invis]
    55[shape=octagon]
    55 -- 0[style=invis]
    55 -- 86
    86 -- 72
    o[style=invis]
    86 -- o[style=invis]

    a0[style=invis]
    a13[label=13]
    a27[label=27]
    a32[label=32]
    a41[label=41]
    a55[label=55]
    a72[label=72]
    a86[label=86]

    a41 -- a27
    a27 -- a13
    a27 -- a32
    a41 -- a72
    a72 -- a55
    a72 -- a86
}
\end{dot2tex}

\item 69:
\begin{dot2tex}[autosize]
graph G {
    a0[style=invis]
    a13[label=13]
    a27[label=27]
    a32[label=32]
    a41[label=41]
    a55[label=55]
    a69[label=69]
    a72[label=72]
    a86[label=86]

    a41 -- a27
    a27 -- a13
    a27 -- a32
    a41 -- a72
    a72 -- a55
    a72 -- a86
    a55 -- a0[style=invis]
    a55 -- a69
}
\end{dot2tex}

\item Loeschen 72: Der rechte teilbaum ist hoerer, desshalb wird der Vorgaenger von 72, also 69, gewaehlt, um den knoten zu ersetzen
\begin{dot2tex}[autosize]
graph G {
    a0[style=invis]
    a13[label=13]
    a27[label=27]
    a32[label=32]
    a41[label=41]
    a55[label=55]
    a69[label=69]
    a72[label=72]
    a86[label=86]

    a41 -- a27
    a27 -- a13
    a27 -- a32
    a41 -- a72
    a72 -- a55
    a72[label=""]
    a69[shape=rect]
    a72 -- a86
    a55 -- a0[style=invis]
    a55 -- a69

    41 -- 27
    27 -- 13
    27 -- 32
    41 -- 69
    69 -- 55
    69 -- 86
}
\end{dot2tex}
\end{itemize}
\end{enumerate}

\section*{Aufgabe 2}
\section*{Aufgabe 3}

\end{document}
