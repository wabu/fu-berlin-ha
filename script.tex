\documentclass[a4paper,11pt]{book}

%\topmargin-2cm
%\topskip0cm
%\footskip0cm
%\headsep0cm
\parindent-0.1cm
%\oddsidemargin -1.8cm
%\evensidemargin -1.8cm
%\headheight0cm
%\textheight 28cm
%\textwidth 20cm

\author{Daniel Waeber}
\title{Script}
\date{Semester 6}

\usepackage{ucs}
\usepackage[utf8x]{inputenc}
\usepackage{german}
\usepackage{color}

\pagestyle{empty}
\usepackage{makeidx}
\usepackage{amsmath}
\usepackage{amsfonts}
\usepackage{amssymb,euscript}
\usepackage{dsfont}
\usepackage{listings}
\usepackage{enumerate}
\newfont{\Fr}{eufm10}
\newfont{\Sc}{eusm10}
\newfont{\Bb}{msbm10}
\newcommand{\limin}{\lim_{n\rightarrow\infty}}
\newcommand{\limix}{\lim_{x\rightarrow\infty}}
\newcommand{\limun}{\lim_{n\rightarrow -\infty}}
\newcommand{\limux}{\lim_{n\rightarrow -\infty}}
\newcommand{\limx}{\lim_{x\rightarrow x_0}}
\newcommand{\limh}{\lim_{h\rightarrow 0}}
\newcommand{\defi}{\paragraph{Definition:}}
\newcommand{\bew}{\paragraph{Beweis:}}
\newcommand{\satz}{\paragraph{Satz:}}
\newcommand{\bsp}{\paragraph{Beispiel:}}
\newcommand{\lemma}{\paragraph{Lemma:}}
\newcommand{\N}{\mathds{N}}
\newcommand{\Z}{\mathds{Z}}
\newcommand{\Q}{\mathds{Q}}
\newcommand{\R}{\mathds{R}}
\newcommand{\C}{\mathds{C}}
\newcommand{\K}{\mathds{K}}
\newcommand{\A}{\mathds{A}}
\newcommand{\qed}{$\hfill\blacksquare$}
\newcommand{\arsinh}{\operatorname{arsinh} }
\newcommand{\arcosh}{\operatorname{arcosh} }
\newcommand{\wP}{\mathcal{P} }
\newcommand{\gdw}{$\Leftrightarrow$}
\newcommand{\tf}{$\Rightarrow$}
\newcommand{\mgdw}{\Leftrightarrow}
\newcommand{\mtf}{\Rightarrow}
\newcommand{\Bild}{\text{Bild}}
\newcommand{\Kern}{\text{kern}}
\newcommand{\rg}{\text{rg}}
\newcommand{\deff}{\text{deff}}

\newcommand{\alphato}{\underset{\alpha}\to}
\newcommand{\betato}{\underset{\beta}\to}
\newcommand{\etato}{\underset{\eta}\to}
\newcommand{\ito}{\underset{i}\to}
\newcommand{\sto}{\underset{s}\to}
\newcommand{\kto}{\underset{k}\to}
\newcommand{\xto}{\underset{x}\to}


\begin{document}
\title{Hoehere Algorithmik}
\maketitle

\tableofcontents

\chapter{Intro}
\section{Vorraussetzungen}
\begin{itemize}
\item Datenstrukturen\\
	Arrays, Listen, binaere Baeume, B-Baueme, Hashing
\item Sortierverfahren\\
\item Graphenalgorithmen\\
	bfs, dfs, shortest path, minimal spanning
\end{itemize}

\section{Inhalt}
\begin{itemize}
\item grundlegende Begriffe: Algorithmen, Effizienz
\item Paradigmen zum Algorithmenentwurf
\item fortgeschrittene Datenstrukturen
\item Graphenalgorithmen 
	\begin{itemize}
	\item Netzwerkfluss
	\item Matching
	\end{itemize}
\item String-Matching
\item NP-Vollstaendigkeit
\item Aproximationsalgorithmen
\item Alogrithmische Probleme
	\begin{itemize}
	\item Matr. Multiplikation
	\item Zahlen-Multiplikation
	\item Kryptosysteme
	\item DFT
	\end{itemize}
\item Komplexitaetstheorie
\end{itemize}





\chapter{Algorithmen, Effizienz, Berechnungsmodelle}

\paragraph{Algorrithmus:}
	Methode zur Loesung eines Problems
	\\mathematische Formalisierung: Turing-Maschine, RegisterMaschine

\paragraph{Effizienz:}
	im Verbrauch von Ressourcen (Zeit, Platz)

\paragraph{Entwurf:}
	von Alogrithmen, schon Effizienzbetrachtung

\section{Beispile fuer Sortieralgorithmen}

\paragraph{Gegeben} Folge $S$ von Elementen $a_1, a_2, \cdots, a_n$ eines
linear geordneten Universums
\paragraph{Berechne} die Folge S aufsteigend geordnet

\subsection{Bogosort}
	\paragraph{Analyse}
	es gibt $n!$ Permutationen, fuer jede $n-1$ Vergleche,
	insgesamt also $(n-1)n!$ Vergleiche im schlechtesten Fall,
	$\frac{1}{n}(n-1)n!$ im Mittel.
\subsection{Selectionsort}
	\paragraph{Analyse}
	$\sum_{i=1}^{n-1} i = \frac{n(n-1)}{2}$
\subsection{Mergesort}
	\paragraph{Alalyse}
	$C(1) = 0$;
	$C(n)\leq 2 C(\frac{n}{2}) + n \leq
	 \leq 4 C(\frac{n}{4}) + n + n 
	 \leq 2^k C(\frac{n}{2^k} + kn$ wobei $k=\log n$, also
	werden $O(n\log(n))$ Vergleiche benoetigt
\subsection{Quicksort}
	im Durchschnitt: $O(n) = n \log n$
\subsection{montecarlosort}

\paragraph{Lehre}
\begin{itemize}
\item Analyse sehr wichtig
\item Effizienz berechenbar ohne implementierung (Analyse)
\item 1. Designkonzept: Dived and Conquer
\item es gibt Algorithmen, die den Zufall zu Hilfe nehmen
	(probabilistische Algorithmen)
	\begin{itemize}
	\item Las-Vegas (ergebnis sicher, laufzeit unsicher)
	\item Monte-Carlo (ergebnis unsicher)
	\end{itemize}
\end{itemize}

\section{Berechnungsmodelle}
math. Modell fuer Rechner, um den begriff berechenabr, laufzeit und paltzbedarf
exakt definerien zu koennen.

\subsection{Turingmachines}
\subsection{RegisterMashiene}
	\begin{itemize}
	\item $A := B op C$ (op: +-*/)
	\item $A := B$
	\item $(A)$ (ref)
	\item $\text{goto} L$
	\item $\text{ggz}  B,L$
	\item $\text{halt}$
	\end{itemize}
\paragraph{Berechnebarkeit}
\begin{itemize}
\item Entspricht Zeit und Platz auf realem Rechner
\item dort alledings nur beschraenkt grosse Zahlen darstellbar
\end{itemize}

\paragraph{Einheitskostemass}
\subparagraph{Laufzeit}
Anzahl der Schritte der RM. bei dieser
Eingabe, bis sie haelt.

\subparagraph{Platzbedarf}
Anzal der Benutzten Register

\paragraph{Logarithmisches Kostenmass (LKM)}
\subparagraph{Laufzeit}
Laufzeit einer op haengt von Summer der Laengern der
Binaerdarstellung der benutzten Zahlen ab.

\subparagraph{Platzbedarf}

\end{document}
