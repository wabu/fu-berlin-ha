\documentclass[a4paper,10pt]{report}

\topmargin -3cm
%\topskip0cm
%\footskip0cm
%\headsep0cm
\parindent0cm
\oddsidemargin -1cm
\evensidemargin -1cm
\headheight 3cm
\textheight 23cm
\textwidth 18cm

\author{Daniel W\"aber (4049590)}
\title{\"Ubung}

\usepackage{ucs}
\usepackage[utf8x]{inputenc}
\usepackage{german}
\usepackage{color}

\pagestyle{empty}
\usepackage{makeidx}
\usepackage{amsmath}
\usepackage{amsfonts}
\usepackage{amssymb,euscript}
\usepackage{dsfont}
\usepackage{listings}
\usepackage{enumerate}
\newfont{\Fr}{eufm10}
\newfont{\Sc}{eusm10}
\newfont{\Bb}{msbm10}
\newcommand{\limin}{\lim_{n\rightarrow\infty}}
\newcommand{\limix}{\lim_{x\rightarrow\infty}}
\newcommand{\limun}{\lim_{n\rightarrow -\infty}}
\newcommand{\limux}{\lim_{n\rightarrow -\infty}}
\newcommand{\limx}{\lim_{x\rightarrow x_0}}
\newcommand{\limh}{\lim_{h\rightarrow 0}}
\newcommand{\defi}{\paragraph{Definition:}}
\newcommand{\bew}{\paragraph{Beweis:}}
\newcommand{\satz}{\paragraph{Satz:}}
\newcommand{\bsp}{\paragraph{Beispiel:}}
\newcommand{\lemma}{\paragraph{Lemma:}}
\newcommand{\N}{\mathds{N}}
\newcommand{\Z}{\mathds{Z}}
\newcommand{\Q}{\mathds{Q}}
\newcommand{\R}{\mathds{R}}
\newcommand{\C}{\mathds{C}}
\newcommand{\K}{\mathds{K}}
\newcommand{\A}{\mathds{A}}
\newcommand{\qed}{$\hfill\blacksquare$}
\newcommand{\arsinh}{\operatorname{arsinh} }
\newcommand{\arcosh}{\operatorname{arcosh} }
\newcommand{\wP}{\mathcal{P} }
\newcommand{\gdw}{$\Leftrightarrow$}
\newcommand{\tf}{$\Rightarrow$}
\newcommand{\mgdw}{\Leftrightarrow}
\newcommand{\mtf}{\Rightarrow}
\newcommand{\Bild}{\text{Bild}}
\newcommand{\Kern}{\text{kern}}
\newcommand{\rg}{\text{rg}}
\newcommand{\deff}{\text{deff}}

\newcommand{\alphato}{\underset{\alpha}\to}
\newcommand{\betato}{\underset{\beta}\to}
\newcommand{\etato}{\underset{\eta}\to}
\newcommand{\ito}{\underset{i}\to}
\newcommand{\sto}{\underset{s}\to}
\newcommand{\kto}{\underset{k}\to}
\newcommand{\xto}{\underset{x}\to}

\usepackage{fancyhdr}
\pagestyle{fancy}
\lhead{Daniel Waeber\\Michael Kmoch}
\chead{"Ubungsblatt \nr\\\today}
\rhead{HA\\Tutor: Claudia Dieckmann}



\newcommand{\nr}{3}

\begin{document}
\section*{Aufgabe 1}
\begin{enumerate}[(a)]
\item $SELECT(k,S)$
    \begin{enumerate}[1.]
    \item Falls $S={a}$ return $a$
    \item Sonst $a = MEDIAN(S)$
    \item 
    \begin{tabbing}
    Teile $S$ in
    \= $S_1$ (Elemente $<a$) \\
    \> $S_2$ (Elemente $=a$) \\
    \> $S_3$ (Elemente $>a$)
    \end{tabbing}
    \item Falls $|S_1| < k \leq |n - S_3|$ return $a$
    \item Sonst falls $|S_1| \geq k$ return $SELECT(k,S_1)$
    \item Sonst return $SELECT(k-(|S|-|S_3|), S_3)$
    \end{enumerate}

    \paragraph{Laufzeit} \ 

    \begin{eqnarray}
    T(1) &=& 1\\
    T(n) &=& 2\cdot n + T(\frac{n}{2})\\
         &=& 2\cdot n + n + T(\frac{n}{4})\\
         &=& 2\cdot n + n + \frac{n}{2} + T(\frac{n}{8})\\
         &=& 2\cdot \sum_{i=1}^{k} \frac{n}{2^i} + T(\frac{n}{2^k})\\
         &=& 2n\cdot \sum_{i=1}^{\log n} \frac{1}{2^i}\\
         &=& \Omega(n)
    \end{eqnarray}

\item $QUICKSORT(S)$
    \begin{itemize}
    \item Falls $S={a}$ return $a$
    \item Sonst $a=MEDIAN(S)$
    \item
    \begin{tabbing}
    Teile $S$ in
    \= $S_1$ (Elemente $<a$) \\
    \> $S_2$ (Elemente $=a$) \\
    \> $S_3$ (Elemente $>a$)
    \end{tabbing}
    \item Sortiere $S_1$ und $S_2$ rekursiver
    \item return $S_1 S_2 S_3$
    \end{itemize}

    \paragraph{Laufzeit} \ 

    \begin{eqnarray}
    T(1) &=& 1\\
    T(n) &=& 2\cdot n + 2 \cdot T(\frac{n}{2})\\
         &=& 4\cdot n + 4 \cdot T(\frac{n}{4})\\
         &=& 6\cdot n + 8 \cdot T(\frac{n}{8})\\
         &=& 2k\cdot n + 2^k \cdot T(\frac{n}{2^k})\\
         &=& 2 n \cdot \log n + n \cdot T(1)\\
         &=& \Omega(n \cdot \log n)
    \end{eqnarray}

\end{enumerate}

\section*{Aufgabe 2}
\begin{enumerate}[(a)]
\item 
    \begin{eqnarray}
    T(1) &=& 0 \\
    T(n) &=& n + \frac{1}{n} \left( \sum_{i=1}^{k-1} T(n-i) + \sum_{i=k+1}^{n} T(i-1) \right) \\
         &\leq& n + \frac{2}{n} \sum_{i=\lfloor \frac{n}{2} \rfloor}^{n-1} T(i) \\
    \end{eqnarray}

    Behauptung: $T(n) \leq d \cdot n$

    \paragraph{IA} $n=1$: 
    \begin{eqnarray}
        T(1) &=& 1 \\ &\leq& d \cdot n \text{ fuer $d\geq 1$}
    \end{eqnarray}

    \paragraph{IS} $\forall i\leq n-1$ $\to$ $n$

    \begin{eqnarray}
    T(n) 
         &\leq& n + \frac{2}{n} \sum_{i=\lfloor \frac{n}{2} \rfloor}^{n-1} T(i) \\
         &\leq& n + \frac{2d}{n} \sum_{i=\lfloor \frac{n}{2} \rfloor}^{n-1} i \\
         &\leq& n + \frac{2d}{n} \cdot \frac{n(n-1) - \frac{n}{2}(\frac{n}{2}-1)}{2}  \\
         &\leq& n + \frac{2d}{n} \cdot (\frac{3}{8}n^2 + \frac{1}{4} n) \\
         &\leq& n + \frac{3}{4} d n + \frac{1}{2} d) \\
         &=& \left( \frac{3}{4} d + 1 \right) n + \frac{d}{2}
    \end{eqnarray}

\item
    Liegt der Range des zufaellig ausgewaehlten Elementes $a$ zwischen $\frac{1}{3}n$ und $\frac{2}{3}n$,
    ist die Menge maxiamal $\frac{2}{3} n$ gross. Die Wahrscheinlichkeit fuer diesen Fall betraegt $\frac{1}{3}$.
    Sonst hat die rekursiv durchsuchte Menge maximal $n-1$ Elemente.

    Per induktion soll lineare Laufzeit gezeigt werden: 
    
    \paragraph{IB}
    \begin{equation}
    T(n) \leq d \cdot n
    \end{equation}

    \paragraph{IA} $n=1$:
    \begin{equation}
    T(1) = const\leq d \cdot n \text{ fuer $d \geq const$ }
    \end{equation}

    \paragraph{IS} $\leq n-1$ $\to$ $n$

    \begin{eqnarray}
    T(n) &=& n \cdot c + \frac{1}{3} T(\frac{2}{3} n) + \frac{2}{3} T(n-1) \\
         &\leq& n \cdot c + \frac{2}{9} d n + \frac{6}{9} d \cdot (n-1) \\
         &=& n \cdot c + \frac{8}{9} dn - \frac{4}{9} d \\
         &=& n (c + \frac{8}{9} d) - \frac{4}{9} d \\
         &\leq& n (c + \frac{8}{9} d) \\
         \text{fuer $d=9c$: }\\
         &=& n (c + \frac{8}{9} \cdot 9c) = 9 n c = d \cdot n
    \end{eqnarray}
\end{enumerate}

\section*{Aufgabe 3}

\end{document}
