\include{headerueb}
\usepackage{ucs}
\usepackage[utf8x]{inputenc}

\usepackage{dot2texi}
\usepackage{tikz}
\usetikzlibrary{shapes,arrows}
\usetikzlibrary{decorations.pathmorphing}
\usepackage{german}
\usepackage{color}

\pagestyle{empty}
\usepackage{alltt}
\usepackage{makeidx}
\usepackage{amsmath}
\usepackage{amsfonts}
\usepackage{amssymb,euscript}
\usepackage{dsfont}
\usepackage{listings}
\usepackage{enumerate}
\newfont{\Fr}{eufm10}
\newfont{\Sc}{eusm10}
\newfont{\Bb}{msbm10}
\newcommand{\limin}{\lim_{n\rightarrow\infty}}
\newcommand{\limix}{\lim_{x\rightarrow\infty}}
\newcommand{\limun}{\lim_{n\rightarrow -\infty}}
\newcommand{\limux}{\lim_{n\rightarrow -\infty}}
\newcommand{\limx}{\lim_{x\rightarrow x_0}}
\newcommand{\limh}{\lim_{h\rightarrow 0}}
\newcommand{\defi}{\paragraph{Definition:}}
\newcommand{\bew}{\paragraph{Beweis:}}
\newcommand{\satz}{\paragraph{Satz:}}
\newcommand{\bsp}{\paragraph{Beispiel:}}
\newcommand{\lemma}{\paragraph{Lemma:}}
\newcommand{\N}{\mathds{N}}
\newcommand{\Z}{\mathds{Z}}
\newcommand{\Q}{\mathds{Q}}
\newcommand{\R}{\mathds{R}}
\newcommand{\C}{\mathds{C}}
\newcommand{\K}{\mathds{K}}
\newcommand{\A}{\mathds{A}}
\newcommand{\qed}{$\hfill\blacksquare$}
\newcommand{\arsinh}{\operatorname{arsinh} }
\newcommand{\arcosh}{\operatorname{arcosh} }
\newcommand{\wP}{\mathcal{P} }
\newcommand{\gdw}{$\Leftrightarrow$}
\newcommand{\tf}{$\Rightarrow$}
\newcommand{\mgdw}{\Leftrightarrow}
\newcommand{\mtf}{\Rightarrow}
\newcommand{\Bild}{\text{Bild}}
\newcommand{\Kern}{\text{kern}}
\newcommand{\rg}{\text{rg}}
\newcommand{\deff}{\text{deff}}

\newcommand{\alphato}{\underset{\alpha}\to}
\newcommand{\betato}{\underset{\beta}\to}
\newcommand{\etato}{\underset{\eta}\to}
\newcommand{\ito}{\underset{i}\to}
\newcommand{\sto}{\underset{s}\to}
\newcommand{\kto}{\underset{k}\to}
\newcommand{\xto}{\underset{x}\to}

\include{info}


\newcommand{\nr}{9}

\begin{document}
\section*{Aufgabe 1}
\paragraph{a)}
Sei $f(x) = c^T \cdot x = c_1 x_1 + c_2 x_2 + \cdots c_n x_n$ das Optimierunsproblem, mit Nebenbedingunen\\
    $a_{1,i_1}  x_{i_1} + a_{1,i_2} x_{i_2} + \cdots + a_{1,i_k} x_{i_k} \begin{matrix}\leq\\=\\\geq\end{matrix}b_1$, \\
    $a_{2,i_1} x_{i_1} + \cdots$, \\\ldots

\begin{itemize}
\item Handelt es sich um ein Maxierungsproblem, so kann statt dessen $\tilde{f}(x) = (-c)^T x$ minimiert werden.
\item Nebenbedingungen der From $a_{i_1} x_{i_1} + a_{i_2} x_{i_2} + \cdots + a_{i_k} i_{i_k} = b$ koennen
    durch $a_{i_1} x_{i_1} + a_{i_2} x_{i_2} + \cdots + a_{i_k} i_{i_k} \leq b$ und 
    $a_{i_1} x_{i_1} + a_{i_2} x_{i_2} + \cdots + a_{i_k} i_{i_k} \geq b$ und 
    ersetzt werden.
\item Nebenbedingungen der Form $a_{i_1} x_{i_1} + a_{i_2} x_{i_2} + \cdots + a_{i_k} i_{i_k} \geq b$ koennen
    durch $-a_{i_1} x_{i_1} - a_{i_2} x_{i_2} - \cdots - a_{i_k} i_{i_k} \leq b$ ersetzt werden,
    da fuer jede Belegung $x_{i_j}$ die erste Form genau dann gilt, wenn auch die Zweite gilt.
\item Vorzeichenbedinungen der Form $x_i \leq 0$, koennen durch $x_i' \geq 0$ ersetzt werden, 
    wobei $x_i$ durch $-x'_i$ ausgetascht werden muss.
    Wurde in der unveraenderten From eine Belegung fuer $x_i \leq 0$ gefunden, so gelten nun fuer $x_i' = -x_i$ 
    alle Nebenbedinugen und $x_i \geq 0$.
\item Variablen $x_i$ ohne Vorzeichenbedinung werden durch $(x_i^+ - x_i^-)$ ersetzt, mit der Bedingung 
    Fuer jeden Belgung $x_i \geq 0$, kann $x_i^+=x_i$, $x_i^-=0$ gesetzt werden, 
    fuer jeden Belgung $x_i \leq 0$, kann $x_i^+=0$, $x_i^-=-x_i$ gesetzt werden, sodass nun $(x_i^+ - x_i^-) = x_i$, 
    $x_i^+ \geq 0$ und  $x_i^- \geq 0$
\end{itemize}

\paragraph{b)}
Ist die maximiernde Zielfunktion von der Form $\min(c^T x, d^T x)$

\section*{Aufgabe 2}
\section*{Aufgabe 3}
\paragraph{a)}

\paragraph{b)}
Zu minimieren ist die Laene des kuerzesten Weges $f(x) = x_t$

Fuer jede Kante $(u,v) \in E$ muessen folgende Nebenbedinungen gelten:
$x_v \leq x_u + c(u,v)$. Zusaetzlich gilt $x_s = 0$

Sei $s, v_1, v_2, \cdots, v_k, t$ ein kuerzest Weg von $s$ nach $t$. Zusaetalich bezeichne
$l_v$ die Lanege des kuerzesten Weges von $s$ nach $v$.

Sei nun $(x_v)_{v\in V}$ eine Belegung mit minimalem $f(x) = x_t$.

\paragraph{Anahme:} $l_t \neq x_t$

\paragraph{1. Fall:} $l_t \geq x_t$:

\begin{eqnarray}
    l_t &=&         c(s,v_1) + c(v_1,v_2) + \cdots + c(v_k,t) \\
      &\geq& x_{v_1} - x_s + c(v_1,v_2) + \cdots + c(v_k,t) \\
      &=&    x_{v_1} + c(v_1,v_2) + \cdots + c(v_k,t) \\
      &\geq& x_{v_1} + x_{v_2} - x_{v_1} + c(v_2,v_3) + \cdots + c(v_k,t) \\
      &=&              x_{v_2}           + c(v_2,v_3) + \cdots + c(v_k,t) \\
      &\geq&           x_{v_3}           + c(v_3,v_4) + \cdots + c(v_k,t) \\
      &\geq& \cdots \\
      &\geq& x_t
\end{eqnarray}
Also kann dieser Fall nicht auftreten.

\paragraph{2. Fall:} $l_t \leq x_t$

Anhand des kuerzesten weges kann eine Belegung gefunden werden, fuer die die Nebenbedungungen gelten: 
\begin{eqnarray}
    x_{v_1}' &=& c(s,v_1) = l_{v_1}\\
    x_{v_2}' &=& x_{v_1}' + c(v_1,v_2) = l_{v_2}\\
    x_{v_3}' &=& x_{v_2}' + c(v_2,v_3) = l_{v_3}\\
    \cdots  &=&  \cdots\\
    x_{v_k}' &=& x_{v_{k-1}}' + c(v_{k-1},v_k) = l_{v_k}\\
    x_t' &=&  x_{v_k}' + c(v_{k}, t) = l_t\\
\end{eqnarray}

Da $l_v$ ein kuerzester Weg von $s$ nach $v$, gilt fuer alle $u \in V$: $l_v \leq l_u + c(u,v)$, 
also auch $x_v' \leq x_u' + c(u,v)$. Also wurde ein Belegung gefunden, fuer die gilt
$x_t' \leq x_t$, also war $x_t$ nicht minimal, was ein Wiederspruch zur Minimierung von $f(x) = x_t$ darstellt.

Somit muss fuer eine minimale Loesung gelten dass $x_t = l_t$, die lanege des kuerzest Weges, ist.

\end{document}
