\documentclass[a4paper,10pt]{report}

\topmargin -3cm
%\topskip0cm
%\footskip0cm
%\headsep0cm
\parindent0cm
\oddsidemargin -1cm
\evensidemargin -1cm
\headheight 3cm
\textheight 23cm
\textwidth 18cm

\author{Daniel W\"aber (4049590)}
\title{\"Ubung}

\usepackage{ucs}
\usepackage[utf8x]{inputenc}
\usepackage{german}
\usepackage{color}

\pagestyle{empty}
\usepackage{makeidx}
\usepackage{amsmath}
\usepackage{amsfonts}
\usepackage{amssymb,euscript}
\usepackage{dsfont}
\usepackage{listings}
\usepackage{enumerate}
\newfont{\Fr}{eufm10}
\newfont{\Sc}{eusm10}
\newfont{\Bb}{msbm10}
\newcommand{\limin}{\lim_{n\rightarrow\infty}}
\newcommand{\limix}{\lim_{x\rightarrow\infty}}
\newcommand{\limun}{\lim_{n\rightarrow -\infty}}
\newcommand{\limux}{\lim_{n\rightarrow -\infty}}
\newcommand{\limx}{\lim_{x\rightarrow x_0}}
\newcommand{\limh}{\lim_{h\rightarrow 0}}
\newcommand{\defi}{\paragraph{Definition:}}
\newcommand{\bew}{\paragraph{Beweis:}}
\newcommand{\satz}{\paragraph{Satz:}}
\newcommand{\bsp}{\paragraph{Beispiel:}}
\newcommand{\lemma}{\paragraph{Lemma:}}
\newcommand{\N}{\mathds{N}}
\newcommand{\Z}{\mathds{Z}}
\newcommand{\Q}{\mathds{Q}}
\newcommand{\R}{\mathds{R}}
\newcommand{\C}{\mathds{C}}
\newcommand{\K}{\mathds{K}}
\newcommand{\A}{\mathds{A}}
\newcommand{\qed}{$\hfill\blacksquare$}
\newcommand{\arsinh}{\operatorname{arsinh} }
\newcommand{\arcosh}{\operatorname{arcosh} }
\newcommand{\wP}{\mathcal{P} }
\newcommand{\gdw}{$\Leftrightarrow$}
\newcommand{\tf}{$\Rightarrow$}
\newcommand{\mgdw}{\Leftrightarrow}
\newcommand{\mtf}{\Rightarrow}
\newcommand{\Bild}{\text{Bild}}
\newcommand{\Kern}{\text{kern}}
\newcommand{\rg}{\text{rg}}
\newcommand{\deff}{\text{deff}}

\newcommand{\alphato}{\underset{\alpha}\to}
\newcommand{\betato}{\underset{\beta}\to}
\newcommand{\etato}{\underset{\eta}\to}
\newcommand{\ito}{\underset{i}\to}
\newcommand{\sto}{\underset{s}\to}
\newcommand{\kto}{\underset{k}\to}
\newcommand{\xto}{\underset{x}\to}

\usepackage{fancyhdr}
\pagestyle{fancy}
\lhead{Daniel Waeber\\Michael Kmoch}
\chead{"Ubungsblatt \nr\\\today}
\rhead{HA\\Tutor: Claudia Dieckmann}



\newcommand{\nr}{8}

\begin{document}
\section*{Aufgabe 1}

\paragraph{IA:} $k=1$

Fuer jede Kante in dem Graphen existiert jeweils ein Weg der Laenge $1$. In der der Adjuszentzmatrix $A$
steht fuer die Kante eine $1$, sodass $A^1$ auch die Anzahl der Wege angiebt.

\paragraph{IS:} $k \to k+1$

Nach vorraussetzung gibt $A^k_{i,j}$ die Anzahl der Wege der Laenge $k$ von $i$ nach $j$ an.

Daraus kann die Anzahl der Wege von $i$ nach $j$ der Laenge $k+1$ berechnet werden,
indem fuer jeden der Knoten $v \in V$ ueberprueft wird, ob ein Weg von $i$ nach $v$ und eine Kante 
von $v$ nach $j$ exisitert:

\begin{eqnarray}
A^{k+1}_{i,j} &=& \sum_{z \in V} \begin{cases} A^k_{i,z} & \text{falls $A^k_{z,j}$} \\0 & \text{sonst}  \end{cases}\\
        &=& \sum_{z \in V} A^k_{i,z} \cdot A^k_{z,j}
\end{eqnarray}

Dies entspricht also der Matrixmultiplikation von $A^k\cdot A = A^{k+1}$ \qed


\paragraph{}
Wird die Addition durch $\lor$ ersetzt, wird nicht die Anzahl der Wege gezaehlt, es wird nur bestimmt, ob ein Weg
der Laenge $k$ existiert.

\section*{Aufgabe 2}

\paragraph{Dijkstra}
\begin{alltt}
fuer alle Knoten \(v \in V\)
  \(D[v] := C(s,v)\)
  \(B[v] := \begin{cases} s \text{falls \(s\) adjszent zu \(v\)} \\ nil \text{sonst} \end{cases} \)
\(S := {s}\)
\(D[s] := 0\)

while \(V \setminus S = \emptyset\) do
    waehle den Knoten \(w \in V \ S\) mit minimalem \(D[w]\)
    \(S := S \cup \{w\}\)
    for each \(u \in V \setminus S\), \(u\) adjazent zu \(w\) do
      if \(D[w]+C(w, u) \leq D[u] \)
         \(D[u] = D[w]+C(w, u)\)
         \(B[u] = w\)
\end{alltt}

$B[u]$ enthaelt nun den umgekehrten Pfad zu dem Startknoten:
\begin{alltt}
outputPath\((t)\)
   \(p = B[t]\)
   outputPaht\((p)\)
   output\((p)\)
\end{alltt}


\paragraph{Floyd Warshall} $ $

Es wird im Algoritmus $B_{ij}$ aufgebaut, das angiebt, welcher Knoten benutzt wurde, 
um den kuerzesten Weg von $i$ nach $j$ zu bestimmen. Wird die Kante $i,j$ benutzt,
ist der Wert von $B_{ij} = \to$, wird kein Wege von $i$ nach $j$ gefunden, ist der Wert $\bot$
\begin{alltt}
for \(i = 1 \cdots n \) do
  for \(j = 1 \cdots n \) do
    \(d\sp{(0)}\sb{ij} = \begin{cases} c(i,j) \text{falls \((i,j) \in E\)} \\ \infty \text{sonst}\end{cases}\)
    \(B\sb{ij} = \begin{cases} \to \text{falls \((i,j) \in E\)} \\ \bot \text{sonst}\end{cases}\)
  end for
end for
for \(k = 1 \cdots n \) do
    for \(i = 1 \cdots n \) do
      for \(j = 1 \cdots n \) do
        \(d\sb1 = d\sp{(k-1)}\sb{ij}\)
        \(d\sb2 = d\sp{(k-1)}\sb{ik} + d\sp{(k-1)}\sb{kj}\)
        if \(d\sb1 \leq d\sb2\)
          \(d\sp{(k)}\sb{ij} = d\sb_1 \)
        else
          \(d\sp{(k)}\sb{ij} = d\sb_2 \)
          \(B\sb{ij} = \to k\)
        end if
     end for
  end for
end for
\end{alltt}

Nun koennen die Wege rekursive aus $B$ bestimmt werden:

\begin{alltt}
outputPath\((i,j,B)\)
   if \(B\sb{ij} = \bot\)
     error "no path from \(i\) to \(j\)"
   if \(B\sb{ij} = \to\)
     output \((i,j)\)
   outputPath\((i,B\sp{ij},B)\)
   outputPath\((B\sp{ij},j,B)\)
\end{alltt}

\end{document}
